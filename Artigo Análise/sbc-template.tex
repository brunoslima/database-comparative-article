\documentclass[12pt]{article}

\usepackage{sbc-template}
\usepackage{graphicx,url}
\usepackage[utf8]{inputenc}
\usepackage[brazil]{babel}

     
\sloppy

\title{Análise Comparativa de Desempenho em Drivers para MongoDB em Aplicações Node.js}

\author{Leandro Ungari Cayres, Ronaldo Celso Messias Correia }


\address{Faculdade de Ciências e Tecnologia -- Universidade Estadual Paulista (UNESP)\\
  Presidente Prudente -- SP -- Brazil
  \email{\{leandro.ungari,ronaldo.correia\}@unesp.br}
}

\begin{document} 

\maketitle

\begin{abstract}
  This meta-paper describes the style to be used in articles and short papers
  for SBC conferences. For papers in English, you should add just an abstract
  while for the papers in Portuguese, we also ask for an abstract in
  Portuguese (``resumo''). In both cases, abstracts should not have more than
  10 lines and must be in the first page of the paper.
\end{abstract}
     
\begin{resumo} 
  Este meta-artigo descreve o estilo a ser usado na confecção de artigos e
  resumos de artigos para publicação nos anais das conferências organizadas
  pela SBC. É solicitada a escrita de resumo e abstract apenas para os artigos
  escritos em português. Artigos em inglês deverão apresentar apenas abstract.
  Nos dois casos, o autor deve tomar cuidado para que o resumo (e o abstract)
  não ultrapassem 10 linhas cada, sendo que ambos devem estar na primeira
  página do artigo.
\end{resumo}


\section{Introdução}
Nos últimos anos, o crescimento no volume de dados mudou a utilização desses por empresas e organizações; tais dados, inicialmente considerados agentes passivos, relacionados às regras de negócio empresarial; tornaram-se potenciais oportunidades de lucro através da análise de informações, e consequentemente, do conhecimento presente no conjunto de dados.

Essa crescente quantidade de dados, chamado de~\emph{Big Data}, não somente requer maior espaço de armazenamento, mas uma mudança em sua organização com base em cada contexto, considerando características como volume, variedade, velocidade e valores~\cite{ward2013undefined}.
A arquitetura dos tradicionais bancos de dados relacionais, baseada no modelo ACID (\textit{atomicity},~\textit{consistency},~\textit{isolation} e~\textit{durability}), não se encaixa de modo adequado em ambientes de~\emph{big data}, principalmente devido a forte consistência dos dados e a garantia de integridade dos relacionamentos. 

Nesse cenário surgem os banco de dados NoSQL (``Not only SQL'') provendo maior flexibilidade estrutural, suporte a replicação e consistência eventual consistência seguindo o critério BASE (\textit{basic},~\textit{availability},~\textit{soft-state} e~\textit{eventual consistency})~\cite{han2011survey}. 
Nos últimos anos, a popularidade dos banco de dados não-relacionais tem crescido~\cite{cooper2010benchmarking,edlich2015nosql} proporcionando diversas soluções conforme características dos dados cada aplicação.

Diferentemente do modelo relacional, os bancos NoSQL não possuem uma linguagem comum entre eles que permita a realização de operações, como o SQL (\emph{Structured Query Language}), desse modo, cada banco provê uma interface nativa para manipulação dos dados. 
Contudo, uso de chamadas de sistema para execução de comandos dessas interfaces não é reconhecida como uma boa prática, que pode incorrer em diversos problemas. 

De modo a contornar essa situação, para os diferentes ambientes de desenvolvimento e linguagens de programação,~\emph{drivers} tem sido desenvolvidos de modo a viabilizar a execução dos comandos no banco de dados. 
Em muitas situações, a decisão de qual combinação entre banco de dados não-relacional e~\emph{driver} a ser empregada pode ser um problema, devido a variedade de possibilidades, assim como o desconhecimento dos pontos positivos e negativos de cada solução.

Neste artigo, por meio de um estudo comparativo, busca-se avaliar as duas principais soluções de~\emph{drivers} para o MongoDB~\footnote{https://www.mongodb.com/}, MongoClient~\footnote{https://mongodb.github.io/node-mongodb-native/} e Mongoose~\footnote{https://mongoosejs.com/}, para ambiente de aplicação Node.js. 
A escolha do banco de dados MongoDB ocorreu devido a sua enorme popularidade recente, empregado em diversas aplicações e linhas de pesquisa; assim como sua integração com o ambiente Node.js, em que ambos utilizam a linguagem de programação JavaScript, de modo a proporcionar um sistema uniforme desde o "cliente-servidor-banco de dados".

falar sobre a escolha do node js, falando que é uma tecnologia crescente.

OBJETIVO - Quais drivers são melhores para cada operação do CRUD?


\section{Banco de Dados Não-Relacional}


\subsection{MongoDB}


\subsubsection{MongoClient}

\subsubsection{Mongoose}



\section{Trabalhos Relacionados} 
\label{section:relacionados}

The first page must display the paper title, the name and address of the
authors, the abstract in English and ``resumo'' in Portuguese (``resumos'' are
required only for papers written in Portuguese). The title must be centered
over the whole page, in 16 point boldface font and with 12 points of space
before itself. Author names must be centered in 12 point font, bold, all of
them disposed in the same line, separated by commas and with 12 points of
space after the title. Addresses must be centered in 12 point font, also with
12 points of space after the authors' names. E-mail addresses should be
written using font Courier New, 10 point nominal size, with 6 points of space
before and 6 points of space after.

The abstract and ``resumo'' (if is the case) must be in 12 point Times font,
indented 0.8cm on both sides. The word \textbf{Abstract} and \textbf{Resumo},
should be written in boldface and must precede the text.

\section{Estudo Comparativo}

\subsection{Dataset}

In some conferences, the papers are published on CD-ROM while only the
abstract is published in the printed Proceedings. In this case, authors are
invited to prepare two final versions of the paper. One, complete, to be
published on the CD and the other, containing only the first page, with
abstract and ``resumo'' (for papers in Portuguese).

\subsection{Análise de Pesquisa}
des

\textbf{Q1} --~\emph{O tamanho médio dos registros impacta de modo relevantes quanto ao uso de memória nas operações de CRUD?}

\textbf{Q2} -- O tamanho médio dos registros pode influenciar no uso de CPU nas operações de CRUD? 

\textbf{Q3} -- O tamanho médio dos registros impacta no tempo de execução de cada uma das operações de CRUD?


\section{Sections and Paragraphs}

Section titles must be in boldface, 13pt, flush left. There should be an extra
12 pt of space before each title. Section numbering is optional. The first
paragraph of each section should not be indented, while the first lines of
subsequent paragraphs should be indented by 1.27 cm.

%\begin{figure}[ht]
%\centering
%\includegraphics[width=.5\textwidth]{fig1.jpg}
%\caption{A typical figure}
%\label{fig:exampleFig1}
%\end{figure}

In tables, try to avoid the use of colored or shaded backgrounds, and avoid
thick, doubled, or unnecessary framing lines. When reporting empirical data,
do not use more decimal digits than warranted by their precision and
reproducibility. Table caption must be placed before the table (see Table 1)
and the font used must also be Helvetica, 10 point, boldface, with 6 points of
space before and after each caption.

%\begin{table}[ht]
%\centering
%\caption{Variables to be considered on the evaluation of interaction
%  techniques}
%\label{tab:exTable1}
%\includegraphics[width=.7\textwidth]{table.jpg}
%\end{table}


\section{Referências Bibliográficas}

\bibliographystyle{sbc}
\bibliography{sbc-template}

\end{document}
