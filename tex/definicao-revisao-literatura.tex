\documentclass[a4paper,12pt,onecolumn,titlepage]{article}

\usepackage{natbib}
\usepackage[T1]{fontenc}
\usepackage[utf8]{inputenc}
%\usepackage[latin1]{inputenc}               %%% acentuação
\usepackage[brazil]{babel}
%\let\cite=\citep
\usepackage{pslatex} %Para fonte Times
\usepackage{microtype} %Para melhorar justificação
\usepackage{graphicx}
\graphicspath{{figuras/}}
\usepackage{geometry}
\geometry{
	left=3cm,
	top=3cm,
	right=2cm,
	bottom=2cm
}
\usepackage{float}
\usepackage{hyperref} %Permite inserir url
\usepackage{setspace} %Permite definir espaçamento
%\usepackage{indentfirst} % Indenta o primeiro parágrafo de cada seção.

\usepackage{amsmath}
\usepackage{multirow}
%\usepackage[dvipsnames,svgnames,x11names]{xcolor}
%\definecolor{grey}{RGB}{221,221,221} %define cor cinza.

%\usepackage{tocbibind}
%\usepackage[alf]{abntex2cite}

%%%%%%%%%%%%%%%%%%%%%%%%%%%%%%%%%%%%%%%%%%%%%%%%%%%%%%%%%%%%%%%%%%%%%%%%%%%%%%%%%%%%%%%%%%%%%%%%%%%%%%%%%%%%%%%%%%%%%%%%%%%%%%%%%%%%%%%%%%%%%%%%%%%%%%%%%%%%%%%%%%

\begin{document} %Inicio do documento

%%%%%%%%%%%%%%%%%%%%%%%%%%%%%%%%%%%%%%%%%%%%%%%%%%%%%%%%%%%%%%%%%%%%%%%%%%%%%%%%%%%%%%%%%%%%%%%%%%%%%%%%%%%%%%%%%%%%%%%%%%%%%%%%%%%%%%%%%%%%%%%%%%%%%%%%%%%%%%%%%%

\begin{titlepage} %Capa	
	\vfill
	\begin{center}	
		{\large \uppercase{Universidade Estadual Paulista ``Júlio de Mesquita Filho''}} \\[0.5cm]
		{\large Faculdade de Ciências e Tecnologia -- Campus de Presidente Prudente} \\[0.5cm]					
		{\large Programa de Pós-Graduação em Ciência da Computação} \\[5cm]
		{\Large \textbf{Definir posteriormente}}\\[6cm]
	\hspace{.45\textwidth}
	\begin{minipage}{.5\textwidth}
		\small Definição e Revisão Bibliográfica do Trabalho da disciplina de Banco de Dados do Programa de Pós-Graduação em Ciência da Computação da Universidade Estadual Paulista.\\[0.5cm]
	\end{minipage}
	\vfill
	\vspace{2.5cm}
	\small Bruno Santos de Lima\\
	\small Leandro Ungari Cayres\\[1cm]
	\small Presidente Prudente - SP\\
	\small Abril - 2019	
	\end{center}	
\end{titlepage}
%Fim da capa

%%%%%%%%%%%%%%%%%%%%%%%%%%%%%%%%%%%%%%%%%%%%%%%%%%%%%%%%%%%%%%%%%%%%%%%%%%%%%%%%%%%%%%%%%%%%%%%%%%%%%%%%%%%%%%%%%%%%%%%%%%%%%%%%%%%%%%%%%%%%%%%%%%%%%%%%%%%%%%%%%%

\onehalfspacing %Espaçamento entre linhas 1.5
%\doublespacing %Espaçameno duplo entre linhas (exigência da FAPESP)

\setcounter{page}{2}
%\tableofcontents %Sumario
%\newpage

%%%%%%%%%%%%%%%%%%%%%%%%%%%%%%%%%%%%%%%%%%%%%%%%%%%%%%%%%%%%%%%%%%%%%%%%%%%%%%%%%%%%%%%%%%%%%%%%%%%%%%%%%%%%%%%%%%%%%%%%%%%%%%%%%%%%%%%%%%%%%%%%%%%%%%%%%%%%%%%%%%

\setcounter{page}{3}

\section{Informações gerais}
\label{sec:informacoes-gerais}

\subsection{Título}
\label{subsec:titulo}

\begin{itemize}

\item Definir posteriormente...

%\item Uma revisão sistemática da literatura sobre o panorama dos estudos acerca de \textit{Self-Admitted Technical Debt}.

%\item O panorama dos estudos acerca de \textit{Self-Admitted Technical Debt}: Uma revisão sistemática da literatura.
\end{itemize}

\subsection{Pesquisadores}
\label{subsec:pesquisadores}

\begin{enumerate}
\item \textbf{Nome:} Bruno Santos de Lima\\
	  \textbf{E-mail:} bruno.s.lima@unesp.br\\
	  \textbf{Titulação:} Bacharel\\
	  \textbf{Instituição:} Universidade Estadual Paulista - Unesp\\
	  \textbf{Página Pessoal:} \url{https://brunoslima.github.io/}\\
	  \textbf{Currículo Lattes:} \url{http://lattes.cnpq.br/2119168921461476}
	  
\item \textbf{Nome:} Leandro Ungari Cayres\\
	  \textbf{E-mail:} leandroungari@gmail.com\\
	  \textbf{Titulação:} Bacharel\\
	  \textbf{Instituição:} Universidade Estadual Paulista - Unesp\\
	  %\textbf{Página Pessoal:} \url{https://brunoslima.github.io/}\\
	  \textbf{Currículo Lattes:} \url{http://lattes.cnpq.br/5996829502147029}

\end{enumerate}

\subsection{Objetivos}
\label{subsec:objetivos}

O objetivo do trabalho a ser desenvolvido consiste em realizar um estudo de caso comparando o desempenho de bancos de dados relacionais e não-relacionais efetuando operações CRUD implementados em um ambiente NodeJS.

\subsection{Metodologia}
\label{subsec:metodologia}

Inicialmente selecionamos bases de dados já utilizadas em outros trabalhos presentes na literatura. Essas bases de dados são utilizados em nosso estudo de caso no qual realizamos uma comparação de desempenho entre banco de dados relacionais e não relacionais efetuando operações CRUD em um ambiente NodeJS. As bases selecionadas foram...

Definidas as bases de dados que subsidiam a realização do estudo de caso, o próximo passo foi delimitar quais bancos de dados relacionais e não relacionais seriam utilizados em nosso estudo.

%%%%%%%%%%%%%%%%%%%%%%%%%%%%%%%%%%%%%%%%%%%%%%%%%%%%%%%%%%%%%%%%%%%%%%%%%%%%%%%%%%%%%%%%%%%%%%%%%%%%%%%%%%%%%%%%%%%%%%%%%%%%%%%%%%%%%%%%%%%%%%%%%%%%%%%%%%%%%%%%%%

%Revisão da Literatura
\section{Banco de Dados Relacional}
\label{sec:relacional}

\section{Banco de Dados Não-Relacional}
\label{sec:nao-relacional}

\section{CRUD}
\label{sec:crud}

\section{Trabalhos Relacionados}
\label{sec:relacionados}

\section{Bases de Dados}
\label{sec:bases}

%%%%%%%%%%%%%%%%%%%%%%%%%%%%%%%%%%%%%%%%%%%%%%%%%%%%%%%%%%%%%%%%%%%%%%%%%%%%%%%%%%%%%%%%%%%%%%%%%%%%%%%%%%%%%%%%%%%%%%%%%%%%%%%%%%%%%%%%%%%%%%%%%%%%%%%%%%%%%%%%%%

%REFERÊNCIAS BIBLIOGRÁFICAS
%\clearpage
\renewcommand{\refname}{Bibliografia}
\bibliographystyle{icmc2}
\bibliography{referencias/referencias}

\end{document}